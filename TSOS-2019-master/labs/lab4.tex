\documentclass{article}
\usepackage[utf8]{inputenc}

\title{Lab4}
\author{Andrew DiBella }
\date{September 2020}

\begin{document}

\maketitle

\section{Host and Guest Operating Systems on VMware}
A host operating system is the OS that is in direct communication with the hardware of the machine. The host has access to the kernel and its system calls. It also uses containerized virtualization to partition data and applications on the server. On the other hand, there can be a guest operating system that runs on top of the host OS that acts as a virtual machine. For example you have the option of running windows on a mac computer if you can partition your hard drive. When choosing a host operating system you should ensure that you utilize a lightweight modular infrastructure so you are able to be as close to the kernel and hardware as possible. Both the host OS and the guest OS can be run simultaneously, but the host OS must be initialized first. 

\end{document}

